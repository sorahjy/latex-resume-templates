%%%%%%%%%%%%%%%%%%%%%%%%%%%%%%%%%%%%%%%%%
% Twenty Seconds Resume/CV
% LaTeX Template
% Version 1.0 (14/7/16)
%
% This template has been downloaded from:
% http://www.LaTeXTemplates.com
%
% Original author:
% Carmine Spagnuolo (cspagnuolo@unisa.it) with major modifications by 
% Vel (vel@LaTeXTemplates.com)
%
% License:
% The MIT License (see included LICENSE file)
%
%%%%%%%%%%%%%%%%%%%%%%%%%%%%%%%%%%%%%%%%%

%----------------------------------------------------------------------------------------
%	PACKAGES AND OTHER DOCUMENT CONFIGURATIONS
%----------------------------------------------------------------------------------------

\documentclass[letterpaper,UTF8]{twentysecondcv} % a4paper for A4


%中文支持
\usepackage{xeCJK}
\usepackage{xltxtra}
\XeTeXlinebreaklocale "zh" 
% \XeTeXlinebreakskip = 0pt plus 1pt 

\setmainfont[Mapping=tex-text]{Times New Roman} % rm
\setsansfont[Mapping=tex-text]{Arial}           % sf
\setmonofont{Courier New}                       % tt



% Command for printing skill progress bars
\newcommand\skills{ 
~
	\smartdiagram[bubble diagram]{
        \textbf{\large{算法设计}}\\\textbf{\large{和建模}},
        \textbf{\large{关系型}}\\\textbf{\large{数据库}},
        \textbf{\large{设计模式}},
        \textbf{\large{Latex}}\\\textbf{\large{写论文}},
        \textbf{\large{机器学}}\\\textbf{\large{习算法}},
        \textbf{\large{数据}}\\\textbf{\large{挖掘}},
        \textbf{\large{大数据}}\\\textbf{\large{原理与技术}},
        \textbf{\large{分布式}}\\\textbf{\large{计算}}
    }
}

\interests{{MarkDown/3.5},{C,C++/4.7},{Latex/3.8},{HTML,CSS,JS/4.2},{Python/4.5},{Java/5.5}}

%----------------------------------------------------------------------------------------
%	 PERSONAL INFORMATION
%----------------------------------------------------------------------------------------

% If you don't need one or more of the below, just remove the content leaving the command, e.g. \cvnumberphone{}

\profilepic{img/avatar.jpg} % Profile picture

\cvname{\qquad 黄君扬} % Your name
\cvjobtitle{\qquad \ 本科三年级\ 在读} % Job title/career
\cvlinkedin{https:github.com/sorahjy}
\cvnumberphone{+86 180-1900-2473} % Phone number
\cvsite{https://blog.sorahjy.com} % Personal website
\cvmail{sorahjy@gmail.com} % Email address

%----------------------------------------------------------------------------------------

\begin{document}
\makeprofile % Print the sidebar

%----------------------------------------------------------------------------------------
%	 EDUCATION
%----------------------------------------------------------------------------------------
\section{Education}

\begin{twenty} % Environment for a list with descriptions
	\twentyitem
    	{预计毕业 \\ 2019.7}
        {计算机科学与技术}
        {}
        {上海理工大学 光电信息与计算机工程学院}
        {GPA: 4.03, 排名:1}
	% \twentyitem
 %    	{2009 - 2013}
 %        {BEng., Computer Engineering}
 %        {\href{http://www.unipune.ac.in/}{University of Pune}}
 %        {Pune, Maharashtra, India}
 %        {GPA: 4.0, First Class with Distinction}
	%\twentyitem{<dates>}{<title>}{<organization>}{<location>}{<description>}
\end{twenty}


\section{Awards}
\begin{twenty}
	\twentyitem
    	{2016 - 2018}
        {上海理工大学学习优秀奖学金}
        {\textbf{{\color{violet}{\normalsize{连续四次一等奖}}}}}
        {}
        {}
    \twentyitem
        {2017}
        {上海市奖学金}
        {\textbf{{\color{violet}{\normalsize{市级表彰}}}}}
        {}
        {}
    \twentyitem
        {2017.4}
        {第八届蓝桥杯Java软件开发省赛(上海)}
        {\textbf{{\color{violet}{\normalsize{一等奖}}}}}
        {}
        {}
    \twentyitem
        {2017.5}
        {第八届蓝桥杯Java软件开发全国总决赛}
        {\textbf{{\color{violet}{\normalsize{二等奖}}}}}
        {}
        {}
    \twentyitem
        {2017.11}
        {第42届ACM-ICPC亚洲区域赛(青岛)}
        {\textbf{{\color{violet}{\normalsize{银奖}}}}}
        {}
        {}
    \twentyitem
        {2017.11}
        {2017年APMCM亚太地区大学生数学建模竞赛}
        {\textbf{{\color{violet}{\normalsize{二等奖}}}}}
        {}
        {}
    \twentyitem
        {2018.2}
        {2018年美国大学生数学建模竞赛}
        {\textbf{{\color{violet}{\normalsize{二等奖}}}}}
        {}
        {}
    \twentyitem
        {2018.4}
        {第三届中国高校计算机团体程序设计天梯赛}
        {\textbf{{\color{violet}{\normalsize{上海市特等/全国三等奖}}}}}
        {}
        {}
    \twentyitem
        {2018.4}
        {第九届蓝桥杯Java软件开发上海市}
        {\textbf{{\color{violet}{\normalsize{一等奖}}}}}
        {}
        {}
\end{twenty}

%----------------------------------------------------------------------------------------
%	 EXPERIENCE
%----------------------------------------------------------------------------------------


\section{Experience}

\begin{twenty} % Environment for a list with descriptions
	\twentyitem
    	{2017.11 - \\2018.1}
        {酒店管理系统}
        {}
        {\href{https://github.com/sorahjy/HotelAstolfo}{https://github.com/enihsyou/HotelAstolfo}}
        {个人项目,和其他两位同学一起开发。
        {\begin{itemize}
        \item In a team of two, won \$8,000 in funding from The Hub incubator (CBaSE, University of Guelph) to kick start a startup venture
        \item In a team of three, built a prototype hyperlocal content platform. The goal was to deliver hyperlocal news, events and other content from community organizations to local users
        \item Built hybrid mobile and web apps using Node.js, Ionic, AngularJS and MongoDB. Performed automated testing using Selenium.
        \item Met with city officials, including the Mayor, and university executives to marketing \& business strategies for the platform 
    \end{itemize}}
        }
        
    \twentyitem
   		{2017.12 - \\ 至今}
        {一种便携的集成身份认证方法}
        {}
        { \href{https://github.com/sorahjy/Identity-Authentication-WeAPP}{https://github.com/sorahjy/Identity-Authentication-WeAPP}}
        {个人项目。该项目的理念是以一种平台的方式,将传统的地理定\\
        位,动态密码,人脸识别三项技术相融合,并结合微信用户唯一\\
        的openid,提供多维度、多层次的身份认证服务。
        {\begin{itemize}
        \item 部署方便,成本低廉,无需额外硬件设备。
        \item 认证迅速,容易维护,应用范围广。
        \item 易推广,易扩展。
        \item 采用MVVM模式,部分页面采用了SPA(单页面应用)技术。
    \end{itemize}}
        }


\section{Certificates}

	%\twentyitem{<dates>}{<title>}{<location>}{<description>}
\end{twenty}
\end{document} 
